%% 
%%	This is file 'beamer_sample.tex'
%%	according to an MPIDR's PowerPoint template (?)
%%	
%%	by Eric Naujoks
%%
%%	Problems, bugs and comments to 
%%	naujoks@demogr.mpg.de
%%

%%%%%%%%%%%%%%%%%%%%%%%%%%%%%%%%%%
%%	Praelegomena								%%
%%%%%%%%%%%%%%%%%%%%%%%%%%%%%%%%%%
%%	- Make sure that you use utf8-encoding for all your .tex-files!!! (TeXnicCenter since version 2.0)
%%	- TeXnicCenter update: MPIDR intranet > Hard- & Sortfware > Software > Script and text editors > TeXnicCenter

\documentclass[20pt]{beamer}

\usepackage[ngerman,english]{babel}
\usepackage{tikz}
\usepackage[normalem]{ulem}
\geometry{paperwidth=10in, paperheight=7.5in}
\usepackage{animate}

\usepackage[utf8]{inputenc}

\usepackage[mpidr]{./mpidr/beamerthemeMPIDR}
\usefonttheme{serif}
\newcolumntype{C}[1]{>{\centering\let\newline\\\arraybackslash\hspace{0pt}}m{#1}}

%% Declaring title and author
\title{Time spent and left of transient states in
stationary populations}
\subtitle{Tim Riffe \\ Francisco Villavicncio}		%%

%%	the institute's logo
\renewcommand{\mylogo}{\includegraphics[width=4.7in]{mpidr_logo_colour_en}}
\usepackage{color}
\definecolor{mygray}{rgb}{0.8,0.8,0.8}
\definecolor{yellow}{rgb}{1,1,0}

\defbeamertemplate{description item}{align left}{\insertdescriptionitem\hfill}
%%	should be the very last package to be loaded
\usepackage{hyperref}

%%%%%%%%%%%%%%%%%%%%%%%%%%%%%%%%%%
%%	Beginning of the document		%%
%%%%%%%%%%%%%%%%%%%%%%%%%%%%%%%%%%
\begin{document}

%%	titlepage - fixed frame:
%%	========================

% \begin{frame}
% 	\titlepage
% \end{frame}
\begin{frame}[plain]
	%\titlepage
	\vspace{-4.4cm}
 \centerline{\includegraphics[scale=.165]{beamerstrip3.png}}

	
	\huge
	\vspace{1em}
	
	Time spent and left of transient states in stationary populations\\
	\vspace{1em}
	\large 
	Tim Riffe \& Francisco Villavicencio
\end{frame}
%-------------------


\begin{frame}[plain]
\Large
 \begin{block}{Brouard-Carey equality}
  Under stationarity, the population aged $x$ equals the population with $x$ life left to live.\\ 
 \end{block}
\small{(Brouard, 1989; Vaupel, 2009; Villavicencio \& Riffe, 2016)}
\end{frame}

\begin{frame}[plain]
\Large
\centering
A visual explanation

\only<1>{\hspace{-8em}\includegraphics[width=\textwidth, keepaspectratio]{Figures/LifeLineSeq/chron1.pdf}}
\only<2>{\hspace{-8em}\includegraphics[width=\textwidth, keepaspectratio]{Figures/LifeLineSeq/chron2.pdf}}
\only<3>{\hspace{-8em}\includegraphics[width=\textwidth, keepaspectratio]{Figures/LifeLineSeq/chron3.pdf}}
\only<4>{\hspace{-8em}\includegraphics[width=\textwidth, keepaspectratio]{Figures/LifeLineSeq/chron4.pdf}}
\only<5>{\hspace{-8em}\includegraphics[width=\textwidth, keepaspectratio]{Figures/LifeLineSeq/chron5.pdf}}
\only<6>{\hspace{-8em}\includegraphics[width=\textwidth, keepaspectratio]{Figures/LifeLineSeq/than1.pdf}}
\only<7>{\includegraphics[width=\textwidth, keepaspectratio]{Figures/LifeLineSeq/together.pdf}}
\end{frame}

\begin{frame}[plain]
\includegraphics[width=\textwidth,keepaspectratio]{Figures/BrouardCareyLegos.png}

Brouard-Carey Lego design here:\\
http://www.publishyourdesign.com/design/70442
\end{frame}

% \begin{frame}[plain]
% \centering
% \includegraphics[scale = .5]{Figures/ConjectureBSPS.png}
% \end{frame}
% \begin{frame}[plain]
% \Large
% \begin{center}
% The distribution of
% Time spent = time left in transient states\\ in stationary populations.
% \end{center}
% \end{frame}

\begin{frame}[plain]
\Large
 \begin{block}{Brouard-Carey Symmetry}
  The age distribution is identical to / symmetrical with the time-to-death distribution.
 \end{block}
\end{frame}

\begin{frame}[plain]
\Large
 \begin{block}{Transient Symmetry}
  Within a given \textcolor{red}{state}, the \textcolor{blue}{time-spent} distribution is equal to the \textcolor{blue}{time-to-exit} distribution
 \end{block}
\end{frame}

\begin{frame}[plain]
\Large
 \begin{block}{Transient equality}
  Under stationarity, the probability that a randomly selected individual is in state $s$ and entered $s$ $x$ years ago is equal to the probability of being in state $s$ and exiting in $x$ years.
 \end{block}
\end{frame}

\begin{frame}[plain]
\Large
\centering
Requisites:
\begin{itemize}
\item all vital and state transition schedules fixed
\item no growth (births = deaths)
\end{itemize}
\end{frame}

\begin{frame}[plain]
\Large
\centering
Probabilistic result:
\begin{itemize}
\item The expected age-state structure is frozen.
\item Each potential discrete state trajectory has a fixed probability of
occurring.
\item Same for past and future cohorts.
\end{itemize}
\end{frame}

\begin{frame}[plain]
\Large
\centering
Deterministic result (problematic):
\begin{itemize}
\item The age-state structure is frozen.
\item Each discrete state trajectory occurs for a fixed fraction of a birth
cohort.
\item Same for past and future cohorts.
\end{itemize}
\end{frame}

\begin{frame}[plain]
\Large
\centering
Deterministic result (friendly):
\begin{itemize}
\item The age-state structure is frozen.
\item The same finite set of discrete state trajectories
\item Same for past and future cohorts (all clones).
\end{itemize}
\end{frame}

\begin{frame}[plain]
\begin{overlayarea}{\textwidth}{.4\textheight}
\begin{center}
\only<1>{
$A^{(i)}=\left\{ 0\right\}$\\
\includegraphics[scale = 1]{Figures/SingleLifeAnim1/step0.pdf}\\
$T^{(i)}=\left\{ \tau_1 \right\}$
}
\only<2>{
$A^{(i)}=\left\{0, a_1 \right\}$\\
\includegraphics[scale = 1]{Figures/SingleLifeAnim1/step1.pdf}\\
$T^{(i)}=\left\{ \tau_1 \right\}$
}
\only<3>{
$A^{(i)}=\left\{0, a_1,a_2 \right\}$\\
\includegraphics[scale = 1]{Figures/SingleLifeAnim1/step2.pdf}\\
$T^{(i)}=\left\{ \tau_1,\tau_2 \right\}$
}
\only<4>{
$A^{(i)}=\left\{0, a_1,a_2,a_3 \right\}$\\
\includegraphics[scale =
1]{Figures/SingleLifeAnim1/step3.pdf}\\
$T^{(i)}=\left\{ \tau_1,\tau_2,\tau_3 \right\}$} 
\only<5>{
$A^{(i)}=\left\{0, a_1,a_2,a_3,\ldots, a_{K-1} \right\}$\\
\includegraphics[scale = 1]{Figures/SingleLifeAnim1/step14.pdf}\\
$T^{(i)}=\left\{ \tau_1,\tau_2,\tau_3, \ldots, \tau_{K-1} \right\}$
}
\only<6>{
$A^{(i)}=\left\{0, a_1,a_2,a_3,\ldots, a_{K-1}, a-K \right\}$\\
\includegraphics[scale = 1]{Figures/SingleLifeAnim1/step15.pdf}\\
$T^{(i)}=\left\{ \tau_1,\tau_2,\tau_3, \ldots, \tau_{K-1}, 0 \right\}$
}
\only<7>{
$A^{(i)}=\left\{0, a_1,a_2,a_3,\ldots, a_{K-1}, a-K \right\}$\\
\includegraphics[scale = 1]{Figures/SingleLifeAnim1/stepsmalldelta.pdf}\\
$T^{(i)}=\left\{ \tau_1,\tau_2,\tau_3, \ldots, \tau_{K-1}, 0 \right\}$
}
\end{center}
\end{overlayarea}
\end{frame}

\begin{frame}[plain]
\Large
\begin{center}
Complementarity:\\ \vspace{1em}
Within an individual over time\\  \vspace{.5em} \huge
$A^{(i)} = T^{(i)}$
\end{center}
\end{frame}

% Now we install the new template for the following frames:
\usebackgroundtemplate{%
  \includegraphics[width=\paperwidth,height=\paperheight]{Figures/MountainReflection.jpg}} 
\begin{frame}
\Large
\vspace{1em}
\begin{center}
\includegraphics[scale=.8]{Figures/ReflectionOverlay.pdf}
\end{center}

\vspace{4em}
\begin{flushright}
@timriffe1 \\
@VillavicencioFG
\end{flushright}

\end{frame}
% Now we install another template, effective from now on:

%%%%%%%%%%%%%%%%%%%%%%%%%%%%%%%%%%
%%	End of the document			%%
%%%%%%%%%%%%%%%%%%%%%%%%%%%%%%%%%%
\end{document}










