%% 
%%	This is file 'beamer_sample.tex'
%%	according to an MPIDR's PowerPoint template (?)
%%	
%%	by Eric Naujoks
%%
%%	Problems, bugs and comments to 
%%	naujoks@demogr.mpg.de
%%

%%%%%%%%%%%%%%%%%%%%%%%%%%%%%%%%%%
%%	Praelegomena								%%
%%%%%%%%%%%%%%%%%%%%%%%%%%%%%%%%%%
%%	- Make sure that you use utf8-encoding for all your .tex-files!!! (TeXnicCenter since version 2.0)
%%	- TeXnicCenter update: MPIDR intranet > Hard- & Sortfware > Software > Script and text editors > TeXnicCenter

\documentclass[20pt,usenames,dvipsnames]{beamer}

\usepackage[ngerman,english]{babel}
\usepackage{tikz}
\usepackage[normalem]{ulem}
\geometry{paperwidth=10in, paperheight=7.5in}
\usepackage{animate}

\usepackage[utf8]{inputenc}

\usepackage[mpidr]{./mpidr/beamerthemeMPIDR}
%\usefonttheme{serif}
\newcolumntype{C}[1]{>{\centering\let\newline\\\arraybackslash\hspace{0pt}}m{#1}}
\newcommand*{\QEDA}{\hfill\ensuremath{\blacksquare}}
%% Declaring title and author
\title{Time spent and left of transient states in
stationary populations}
\subtitle{Tim Riffe \\ Francisco Villavicncio}		%%

%%	the institute's logo
\renewcommand{\mylogo}{\includegraphics[width=4.7in]{mpidr_logo_colour_en}}
\usepackage{color}
\definecolor{mygray}{rgb}{0.8,0.8,0.8}
\definecolor{yellow}{rgb}{1,1,0}

\defbeamertemplate{description item}{align left}{\insertdescriptionitem\hfill}
%%	should be the very last package to be loaded
\usepackage{hyperref}

%%%%%%%%%%%%%%%%%%%%%%%%%%%%%%%%%%
%%	Beginning of the document		%%
%%%%%%%%%%%%%%%%%%%%%%%%%%%%%%%%%%
\begin{document}

%%	titlepage - fixed frame:
%%	========================

% \begin{frame}
% 	\titlepage
% \end{frame}
\begin{frame}[plain]
	%\titlepage
	\vspace{-3cm}
 \centerline{\includegraphics[scale=.165]{beamerstrip3.png}}

	
	\huge
	\vspace{1em}
	
	Time spent and left of transient states in stationary populations\\
	\vspace{1em}
	\large 
	Tim Riffe \& Francisco Villavicencio \textcolor{gray}{\& Nicolas Brouard}
\end{frame}
%-------------------


\begin{frame}[plain]
\Large
 \begin{block}{Brouard-Carey equality}
  Under stationarity, the population aged $x$ equals the population with $x$ life left to live.\\ 
 \end{block}
\small{(Brouard, 1989; Vaupel, 2009; Villavicencio \& Riffe, 2016)}
\pause
 \begin{block}{Transient equality}
  Under stationarity, the probability that a randomly selected individual is in
  state \textcolor{OliveGreen}{$\mathbf{s}$} and entered
  \textcolor{OliveGreen}{$\mathbf{s}$} \textcolor{red}{$\mathbf{x}$} years ago
  is equal to the probability of being in state
  \textcolor{OliveGreen}{$\mathbf{s}$} and exiting in
  \textcolor{blue}{$\mathbf{x}$} years.
 \end{block}
\end{frame}

\begin{frame}[plain]
\Large
\centering
Requisites:
\begin{itemize}[<+->]
\item All vital and state transition schedules fixed.
\item No growth (births = deaths).
\item The expected age-state structure is frozen.
\item Each potential discrete state trajectory has a fixed probability of
occurring.
\item Same for past and future cohorts.
\end{itemize}
\end{frame}

%\begin{frame}[plain]
%\Large
%\centering
%Deterministic result (friendly):
%\begin{itemize}[<+->]
%\item The age-state structure is frozen.
%\item The same finite set of discrete state trajectories.
%\item Same for past and future cohorts (all clones).
%\end{itemize}
%\end{frame}

\begin{frame}[plain]
\Large
\centering
Prove for friendly case, generalize to probability case.
\end{frame}

%\begin{frame}[plain]
%\Large
%Step 1
%\begin{overlayarea}{\textwidth}{.4\textheight}
%\begin{center}
%\only<1>{
%$A^{(i)}=\left\{ 0\right\}$\\
%\includegraphics[scale = 1]{Figures/SingleLifeAnim1/step0.pdf}\\
%$T^{(i)}=\left\{ \tau_1 \right\}$
%}
%\only<2>{
%$A^{(i)}=\left\{0, a_1 \right\}$\\
%\includegraphics[scale = 1]{Figures/SingleLifeAnim1/step1.pdf}\\
%$T^{(i)}=\left\{ \tau_1 \right\}$
%}
%\only<3>{
%$A^{(i)}=\left\{0, a_1,a_2 \right\}$\\
%\includegraphics[scale = 1]{Figures/SingleLifeAnim1/step2.pdf}\\
%$T^{(i)}=\left\{ \tau_1,\tau_2 \right\}$
%}
%\only<4>{
%$A^{(i)}=\left\{0, a_1,a_2,a_3 \right\}$\\
%\includegraphics[scale =
%1]{Figures/SingleLifeAnim1/step3.pdf}\\
%$T^{(i)}=\left\{ \tau_1,\tau_2,\tau_3 \right\}$} 
%\only<5>{
%$A^{(i)}=\left\{0, a_1,a_2,a_3,\ldots, a_{K-1} \right\}$\\
%\includegraphics[scale = 1]{Figures/SingleLifeAnim1/step14.pdf}\\
%$T^{(i)}=\left\{ \tau_1,\tau_2,\tau_3, \ldots, \tau_{K-1} \right\}$
%}
%\only<6>{
%$A^{(i)}=\left\{0, a_1,a_2,a_3,\ldots, a_{K-1}, a_{K} \right\}$\\
%\includegraphics[scale = 1]{Figures/SingleLifeAnim1/step15.pdf}\\
%$T^{(i)}=\left\{ \tau_1,\tau_2,\tau_3, \ldots, \tau_{K-1}, 0 \right\}$
%}
%\only<7>{
%$A^{(i)}=\left\{0, a_1,a_2,a_3,\ldots, a_{K-1}, a_{K} \right\}$\\
%\includegraphics[scale = 1]{Figures/SingleLifeAnim1/stepsmalldelta.pdf}\\
%$T^{(i)}=\left\{ \tau_1,\tau_2,\tau_3, \ldots, \tau_{K-1}, 0 \right\}$
%}
%\end{center}
%\end{overlayarea}
%\end{frame}
%%
%\begin{frame}[plain]
%\Large
%\begin{center}
%Complementarity:\\ \vspace{1em}
%Within an individual over time\\  \vspace{.5em} \huge
%\begin{equation}
%A^{(i)} = T^{(i)} \notag
%\end{equation}
%\end{center}
%\end{frame}
%
%\begin{frame}[plain]
%\Large
%\begin{center}
%Complementarity:\\ \vspace{1em}
%for the union of two individuals\\ \vspace{.5em} \huge
%\begin{equation}
%\left\{ A^1 , A^2 \right\} = \left\{ T^1 , T^2 \right\} \notag
%\end{equation}
%\end{center}
%\end{frame}
%
%--------------------------------------------------------
%
%\begin{frame}[plain]
%\vspace{-3em}
%\Large
%\begin{overlayarea}{\textwidth}{.4\textheight}
%\begin{center}
%\only<1>{
%Move to Lexis diagonal\\
%\includegraphics[scale = 1]{Figures/buildstationary/step1.pdf}\\
%}
%\only<2>{
%A clone is born every $\Delta$ time step\\
%\includegraphics[scale = 1]{Figures/buildstationary/step2.pdf}\\
%}
%\only<3>{
%A clone is born every $\Delta$ time step\\
%\includegraphics[scale = 1]{Figures/buildstationary/step3.pdf}\\
%}
%\only<4>{
%A stationary series of clones\\
%\includegraphics[scale = 1]{Figures/buildstationary/step16.pdf}\\
%}
%\only<5>{
%A census in stationary series\\
%\includegraphics[scale = 1]{Figures/buildstationary/spl0.pdf}\\
%}
%\only<6>{
%Complementarity in ``clone'' stationarity\\
%\includegraphics[scale = 1]{Figures/buildstationary/spl1.pdf}\\
%}
%\only<7>{
%Complementarity for some other episode\\
%\includegraphics[scale = 1]{Figures/buildstationary/spl2.pdf}\\
%}
%\only<8>{
%Complementarity for set unions\\
%\includegraphics[scale = 1]{Figures/buildstationary/spl3.pdf}\\
%}
%\end{center}
%\end{overlayarea}
%\end{frame}
%
%\begin{frame}[plain]
%\Large
%\begin{proof}
%By induction, complimentarity for all episodes. This implies equal sets of time
%spent and left values, a.k.a. equal time spent and left distributions.
%\end{proof}
%
%\pause
%\begin{itemize}[<+->]
%  \item Make $\Delta$ smaller if you want.
%  \item Replace ``clones'' w ``fixed fraction''.
%  \item Replace fraction w ``fixed probability''.
%\end{itemize}
%\end{frame}

\begin{frame}[plain]
\Large
\begin{center}
Implications:
\pause
\begin{itemize}[<+->]
  \item Equal time spent-left distributions within states.
  \item Within episode \emph{order}. 
  \item Cumulative over episodes.
  \item Conditional on anything (age, TTD).
  \item Brouard-Carey is degenerate case.
\end{itemize}
\end{center}
\end{frame}
%
%% Now we install the new template for the following frames:
%{
%\usebackgroundtemplate{%
%  \includegraphics[width=\paperwidth,height=\paperheight]{Figures/eberhard-grossgasteiger-410620-unsplash.jpg}} 
%\begin{frame}[plain]
%\tiny
%\flushright
%\vspace{18cm}
%   Photo by Eberhard Grossgasteiger on Unsplash
%\end{frame}
%}
%% Now we install another template, effective from now on:
%\begin{frame}[plain]
%\Large
%\begin{center}
%Simulate
%\begin{itemize}[<+->]
%  \item Take transition matrix from \normalsize{Dudel \& Myrskl\"a (2017)}
%  \item Simulate trajectories; \texttt{rmarkovchain()} in \texttt{markovchain}
%  package.
%  \item Take census in stationary series. 
%  \item Assume observation at half interval.
%  \item Tabulate time spent and left in sampled episodes.
%  \item Compare distributions.
%\end{itemize}
%\end{center}
%\end{frame}
%
%\begin{frame}[plain]
%\Large
%\begin{center}
%\includegraphics[height=\paperheight]{Figures/SimResults.pdf}
%\end{center}
%\end{frame}

{
\usebackgroundtemplate{%
  \includegraphics[width=\paperwidth,height=\paperheight]{Figures/eberhard-grossgasteiger-410620-unsplash.jpg}} 
\begin{frame}
\vspace{9cm}
\Large
\begin{center}
Estimate from the reflection!\\ Thanks! \\
riffe@demogr.mpg.de
\end{center}
\end{frame}
}

%%%%%%%%%%%%%%%%%%%%%%%%%%%%%%%%%%
%%	End of the document			%%
%%%%%%%%%%%%%%%%%%%%%%%%%%%%%%%%%%
\end{document}



% Photo by eberhard grossgasteiger on Unsplash






