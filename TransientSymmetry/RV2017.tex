\documentclass[12pt,oneside,a4paper]{article} % for sharing
\usepackage{apacite}
\usepackage{appendix}
\usepackage{amsmath}
\usepackage{amsthm}

\usepackage{amssymb} % for approx greater than
\usepackage{caption}
\usepackage{placeins} % for \FloatBarrier
\usepackage{graphicx}
\usepackage{subcaption}
\usepackage{longtable}
\usepackage{setspace}
\usepackage{booktabs}
\usepackage{tabularx}
\usepackage{xcolor,colortbl}
\usepackage{chngpage}
\usepackage{natbib}
\bibpunct{(}{)}{,}{a}{}{;} 
\usepackage{url}
\usepackage{nth}
\usepackage{authblk}
\usepackage[most]{tcolorbox}
\usepackage[normalem]{ulem}
\usepackage{amsfonts}
% columns for longtable
\newcolumntype{C}[1]{>{\centering\let\newline\\\arraybackslash\hspace{0pt}}m{#1}}
\newcolumntype{L}[1]{>{\raggedright\let\newline\\\arraybackslash\hspace{0pt}}m{#1}}
\usepackage{arydshln} % Dashed lines in matrices

\usepackage[margin=1in]{geometry}
%\doublespacing % for review

% line numbers to make review easier
%\usepackage{lineno}
%\linenumbers

%\usepackage{soul}% for \st{}

%%%%%%%%%%%%%%%%%%%%%%%%%%%%%%%%%%%%%%%%%%%%%%%%%%%%%%%%%%%%%%%%%%%%%%%%%%%%%%
% for section 4 math environments
\theoremstyle{definition}
\newtheorem{definition}{Definition}[section]
\newtheorem{theorem}{Theorem}[section]
\newtheorem{proposition}{Proposition}[section]
\newtheorem{corollary}{Corollary}[proposition]
\newtheorem{remark}{Remark}[section]

%%%%%%%%%%%%%%%%%%%%%%%%%%%%%%%%%%%%%%%%%%%%%%%%%%%%%%%%%%%%%%%%%%%%%%%%%%%%%%

\newcommand\ackn[1]{%
  \begingroup
  \renewcommand\thefootnote{}\footnote{#1}%
  \addtocounter{footnote}{-1}%
  \endgroup
}

% Affiliations in small font size
\renewcommand\Affilfont{\small}

\defcitealias{HMD}{HMD 2016}

% junk for longtable caption
\AtBeginEnvironment{longtable}{\linespread{1}\selectfont}
\setlength{\LTcapwidth}{\linewidth}

% sort van Raalte properly
% #1: sorting key, #2: prefix for citation, #3: prefix for bibliography
\DeclareRobustCommand{\VAN}[3]{#2} % set up for citation

%%%%%%%%%%%%%%%%%%%%%%%%%%%%%%%
\begin{document}

\title{Symmetry in the forward and backward tenure of transient states in
multistate stationary populations}
%\author{author(s) redacted}
\author[1]{Tim Riffe\thanks{riffe@demogr.mpg.de}}
\author[2,3]{Francisco Villavicencio}
\affil[1]{Max Planck Institute for Demographic Research, Rostock, Germany}
\affil[2]{Max-Planck Odense Center on the Biodemography of Aging, Odense, Denmark}
\affil[3]{Department of Public Health, University of Southern Denmark, Odense, Denmark}

\maketitle

\begin{abstract}

\end{abstract}
Guess what? The Brouard-Carey equality also applies to transient states in
multistate stationary populations. That's even less intuitive than the original
theorem because entries to and exits from states aren't aligned on either end.
This property even holds if the transient state affects mortality, because age
and TTD are beside the point. It'd only be practical if a census were at large.
\section{Title}

\subsection{Subtitle}

Plain text.

\subsection{Another subtitle}

More plain text.


\end{document}
